The first goal for testing was validating that the RRT-connect and the constraint was working as intended. For this purpose a fixed Qinit and Qgoal configuration were used, which are listed below.
\begin{align}
Q_{init} &= (-1.5,-1,1,0,0,0)^{T}  \\
Q_{goal} &=(-2,-0.5,1,-1.8,0,0)^{T}
\end{align}
For this it's known that $Q_{init}$ is inside the constraint radius and the $Q_{goal}$ is outside so it should be possible to find a path and the path is for filling the constraint. This was done by plotting the distance from the tool to the human center in the found path. \\ 
Other than that the overall performance of the algorithm wanted to be testes so by randomly choosing valid $Q_{init}$ and $Q_{goal}$ pairs, by valid means that these two configurations isn't directly in conflict with the constraint and collisions. A histogram of the average number of nodes in the path should then be calculated. \\
Last a histogram of the number of times the main loop is running before a solution was found was calculated to find out how many times really is enough and when we can stop and conclude that there probertly will not be found a solution.

\section{Parameter}
In the algorithm, three parameters were used as shown below. These were chosen based on tests to optimize the speed of the algorithm without making any invalid paths. 
\begin{align}
 \epsilon _{collision checking} &= 0.01  \\
 \epsilon _{node jump} &= 0.5 \\
	K &= 50 
\end{align}

 $\epsilon _{collision checking}$ is the distance of testpoints in the different edges to be validated in configuration space. $\epsilon _{node jump}$ is the max distance between each node in configuration space. K is the max number of times the $RRT\_ CONNECT\_ PLANNER$ will run through before it return a empty path, and was validated as a good number using the result in the result section figure \ref{fig:k}.